\documentclass[main.tex]{subfiles}
\begin{document}
\section{Introduction}

%Dataset:
%https://www.kaggle.com/sohier/London-police-records 
%Use the location data to make clustering of the city areas and the amount of crime in that area.
% 
%Fulfills
%VVV:
%    Volume:
%    1GB data (only London. If we use all of England it would be more)
%     
%    Velocity:
%    2GB new data per month (in all of England)
%     
%    Variety:
%    3 different data sources (.csv files) with different structure (Crime data, Stops and searches, Outcomes)
%    
% 
%This dataset was kindly released by the British Home Office under the Open
%Government License 3.0 at https://data.police.uk/data/.
% 
%Insight:
%We wish to perform clustering to see what patterns are in the data. We expect answers to questions such as:
%
%    What
%    areas contain the most violent crime?
%     
%    What
%    types of crime most often lead to a suspect charged with a crime?
%     
%    Which
%    ethnicities are most often found with drugs when searched?
%     
%
%But surprising insights may also present themselves.
% 
%Justification:
%The insight achieved from this data will be able to aid the London police department in managing their resources throughout the city. 

\textbf{Problem}\\
% Who is the customer?
London police
Civilians
Government

% What value/insight does the system bring?
Resource allocation of police staff among Great Britain
Civilians are able to see "neighbourhoods" with their crime rates which may help to inform where to buy property.
Government are able to inform of possible locations to stay away from.
Government can find patterns in the data in order to gain new knowledge. 

% What insights do we wish to achieve? 
Based on the presented statistics of London crimes, value can be created by identifying trends amongst the data's parameters. Things such as:
\begin{itemize}
    \item Which areas are most prone to crime during holiday seasons?
    \item What is the most prominent crime type of each of London's LSOAs?
    \item Which crime type is most prominent in each area depending on the season?
\end{itemize}

% Visualizations
Heat geo map of crimes
Bar charts (freq)
Contingency tables
\\

Possible methods:
Clustering (lat, long, crime type, outcome, date)
Patterns - market basket analysis (categorical data / transactions) (LSOA, crime type, date, outcome)
e.g. given crime type, LSOA, .., it is it likely that the crime have a certain outcome. 
e.g. a given month and LSOA might lead to a certain crime type
Statistics - e.g. frequency of column x in a given period

%We wish to perform clustering to see what patterns are in the data. We expect answers to questions such as:
What areas contain the most violent crime?
(Cluster location/crime type - look at number of incidents in area, as well as which type of crime is most prominent in each area.)

(Which types of crime are most popular with respect to the time of year? Christmas, Halloween, etc.)

What types of crime most often lead to a suspect charged with a crime?

Which ethnicities are most often found with drugs when searched?

But surprising insights may also present themselves.




\end{document}